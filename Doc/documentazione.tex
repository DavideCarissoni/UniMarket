\documentclass[a4paper,12pt]{article}

% caricamento dei diversi pacchetti
\usepackage{fancyhdr}
\usepackage{graphicx}  % Per inserire le immagini
\usepackage{geometry}
\usepackage{lipsum}
\usepackage{hyperref}  % Per rendere l'indice cliccabile

\geometry{top=2.5cm,bottom=2.5cm,left=2.5cm,right=2.5cm}

% Intestazioni personalizzate
\pagestyle{fancy}
\fancyhf{}  % Pulisce l'intestazione e il piè di pagina
\fancyhead[C]{Progetto di Ingegneria del Software}

\begin{document}

% Titolo
\begin{titlepage}
    \centering
    \vspace*{2cm}
    
    \vspace{1.5cm}
    
    % Titolo del progetto
    \large Documentazione\\
    \Huge
    \textbf{UniMarket}
    
    \vspace{1.5cm}
    
    % Titolo del corso e altri dettagli
    \LARGE
    Progetto di Ingegneria del Software
    
    \vspace{1.5cm}
    \includegraphics[width=0.5\textwidth]{../Media/logo.jpeg}
    
    \vspace{1.5cm}
    % Autori del progetto
    \small
    \textbf{Di Davide Dell'Anno}, \textbf{Davide Carissoni}, \textbf{Francesca Corrente}
    
    \vspace{1.5cm}
    
    % Dati dell'università
    \Large
    Università degli Studi di Bergamo \\
    Facoltà di Ingegneria Informatica \\
    2024-2025
    
\end{titlepage}

% Indice
\renewcommand{\contentsname}{Indice}
\tableofcontents
\newpage


% Capitoli e sottosezioni
\section{Introduzione} %fra
UniMarket è stato progettato con l'obiettivo di semplificare la gestione della spesa per gli studenti.\\  
Si tratta di un'applicazione che consente agli studenti di selezionare facilmente i prodotti desiderati e creare un carrello virtuale, ottimizzando il processo di ricerca e acquisto. \\
L'applicazione non solo semplifica la gestione della spesa quotidiana, ma offre anche funzionalità pensate per le esigenze specifiche degli studenti, come la possibilità di salvare le proprie liste della spesa, monitorare i costi in tempo reale per rispettare il budget mensile e ricevere notifiche su offerte speciali o sconti. \\
UniMarket mira a migliorare l'esperienza di spesa online, offrendo una soluzione pratica ed efficiente per le esigenze quotidiane degli studenti.

\section{Modelli di Processo} %davide
Lo sviluppo del progetto seguirà un processo \textbf{RAD} (Rapid Application Development), con l’obiettivo di realizzare un’applicazione funzionante entro i tempi stabiliti dal \textit{time box}.  
Verrà adottato il \textbf{modello di Kano} per classificare i requisiti del progetto in base alla loro capacità di soddisfare le preferenze del cliente.


\section{Organizzazione del Progetto} %delli
Per questo progetto è stata creata una \textbf{Squadra SWAT}, ovvero un team agile e versatile, composto da membri con competenze complementari, che si adatta 
rapidamente ai cambiamenti del progetto. in questo modo è più facile rispondere prontamente alle esigenze del cliente e dell’ambiente di sviluppo, 
mantenendo un'elevata efficienza nella gestione delle attività. \\
La configurazione del team di sviluppo segue il paradigma \textbf{Agile} e si basa su un modello già collaudato. I ruoli sono stati assegnati come segue: 
\begin{itemize}
    \item \textbf{Scrum Master: X} 
    Ha il compito di coordinare le attività, pianificare i moduli per gli sprint e assicurarsi il rispetto delle scadenze. 
    \item \textbf{Product Owner: Y} 
    Scelto per il suo elevato livello di interazione con il cliente e la profonda conoscenza del progetto.
    \item \textbf{Sviluppatori, Progettisti, Tester: X, Y, Z, W} 
    Vista la piccola grandezza del team, tutti i membri del progetto si occupano di tutte e tre le mansioni.
\end{itemize}

\section{Standard, Linee Guida, Procedure} %fra
È molto importante che all’interno del progetto ogni componente del team coinvolto segua gli standard, le linee guida e le procedure concordate al fine di garantire un risultato facilmente comprensibile a tutti. \\
Gli standard Java Oracle definiscono le regole e le pratiche per la scrittura del codice, assicurando che il software sviluppato rispetti tali linee guida. In questo modo, l'approccio allo sviluppo adottato dal nostro team garantirà uniformità e qualità.\\
Il linguaggio di programmazione scelto per lo sviluppo del software gestionale è Java.\\
Lo sviluppo avviene all'interno dell'IDE Eclipse, che permette la generazione automatica della documentazione tecnica tramite JavaDoc.\\
La Stesura della Documentazione del progetto è scritta utilizzando LaTeX, una scelta che semplifica la gestione della formattazione e il merge delle modifiche, garantendo un output professionale e ben strutturato. Per semplificare e velocizzare il processo di scrittura, è stato utilizzato il plugin LaTeX Workshop.\\
Per garantire una comprensione chiara e accurata del progetto, utilizziamo diagrammi UML (Unified Modelling Language) che permettono di visualizzare in modo intuitivo la struttura del sistema.\\
Per la gestione delle versioni e la condivisione del codice usiamo GitHub, uno strumento essenziale per la collaborazione tra sviluppatori. GitHub consente di tracciare ogni modifica e mantenere il progetto sempre aggiornato. Inoltre, è stato utilizzato anche per la condivisione della documentazione e la gestione delle risorse correlate.\\
Per lo sviluppo dell'interfaccia grafica, il team ha scelto Vaadin.

\section{Attività di Gestione} %davide
Per garantire una gestione efficace delle attività e dello sviluppo del progetto, sono stati definiti i seguenti obiettivi:
\begin{itemize}
    \item \textbf{Incontri settimanali:} monitorare l’avanzamento del lavoro, affrontare eventuali problemi riscontrati, discutere strategie risolutive e valutare la direzione dello sviluppo.
    \item \textbf{Kanban board su GitHub:} tracciare il progresso delle attività, organizzare il lavoro e assegnare incarichi ai membri del team.
    \item \textbf{Gestione del progetto con GitHub:} tenere traccia delle modifiche, monitorare gli aggiornamenti e facilitare la collaborazione.
\end{itemize}

\section{Rischi} %delli
I rischi a cui bisogna prestare più attenzione sono: 
\begin{itemize}
    \item \textbf{Compatibilità dei sistemi utilizzati:} 
    Il progetto viene sviluppato per essere utilizzato su desktop; pur usando maven, che consente il download automatico delle librerie richieste, potrebbero comunque esserci problemi di visualizzazione o calcolo su mobile o altri S.O. desktop. 
    \item \textbf{Limitate quantità e tipologie di prodotto:} 
    UniMarket è indipendente da reali supermarket, non dispone di tante opzioni per ogni genere di prodotto. 
    \item \textbf{Bugs:} 
    Ci potrebbero essere problemi generati dall’uso scorretto e imprevisto del programma da parte del cliente,
    è possibile che alcuni bug agli estremi dell’utilizzo improprio rimangano pur dopo i test. 
\end{itemize}

\section{Personale} % fra
Il team è composto da tre membri:\begin{itemize}
    \item \textbf{Francesca Corrente} 
    \item \textbf{Davide Carissoni} 
    \item \textbf{Davide Dell’anno} 
\end{itemize}
Ogni membro avrà funzioni simili all’interno del team, con ruoli assegnati in base alle esigenze, senza l’intervento di personale esterno.

\section{Metodi e Tecniche} % davide
I metodi e le tecniche adottati per le diverse fasi dello sviluppo sono i seguenti:
\begin{itemize}
    \item \textbf{Pianificazione dei requisiti:} si basa sull'elicitazione e sull'analisi dei requisiti per definire in modo chiaro le funzionalità richieste dall'applicazione.
    \item \textbf{Progettazione dell’applicazione:} prevede la definizione del funzionamento dell'applicazione, delle funzionalità incluse e degli strumenti software da utilizzare.
    \item \textbf{Programmazione:} consiste nello sviluppo del codice sorgente dell’applicazione seguendo le specifiche progettuali.
    \item \textbf{Testing:} include la scrittura e l’esecuzione di test, utilizzando strumenti appositi, per verificare il corretto funzionamento del codice e individuare eventuali anomalie.
\end{itemize}
Queste attività saranno supportate da un’adeguata gestione del controllo di versione e della configurazione dei componenti software, documentando ogni fase con precisione.


\section{Garanzie di Qualità} % delli
Al fine di assicurare la qualità durante il processo di sviluppo e la realizzazione del software,
verranno utilizzati i seguenti criteri:  
\begin{itemize}
    \item \textbf{Correttezza:}  
    Il software deve funzionare come previsto e soddisfare i requisiti specificati, senza errori o comportamenti imprevisti. La correttezza implica che il sistema esegua tutte le operazioni correttamente, restituendo i risultati attesi in ogni condizione di input.

    \item \textbf{Affidabilità:}  
    Il software deve essere stabile e operare in modo continuo e senza interruzioni. Un software affidabile deve resistere a guasti o malfunzionamenti, gestendo situazioni impreviste e mantenendo la sua funzionalità nel tempo, anche sotto carichi elevati o condizioni di stress.

    \item \textbf{Integrità:}  
    L'integrità riguarda la protezione dei dati e il mantenimento della coerenza e correttezza durante le operazioni. Un software con buona integrità assicura che i dati non vengano alterati o danneggiati involontariamente, né durante l'elaborazione né durante la memorizzazione.

    \item \textbf{Usabilità:}  
    Il software deve essere intuitivo e facile da usare per gli utenti finali. Una buona usabilità garantisce che gli utenti possano navigare nell'interfaccia senza difficoltà, riducendo il numero di errori operativi e migliorando l'esperienza complessiva.

    \item \textbf{Manutenibilità:}  
    Un software manutenibile è facilmente aggiornabile e modificabile nel tempo per adattarsi a nuove esigenze o correggere eventuali difetti. Ciò implica che il codice sia chiaro, ben strutturato e documentato, facilitando interventi futuri senza introdurre regressioni o problemi aggiuntivi.
\end{itemize}

\section{Paccheti di lavoro} % fra
Nel team, il lavoro sarà suddiviso e gestito attraverso l'uso della Kanban Board, uno strumento che offre una visione chiara e immediata del flusso di lavoro.\\ La board consente di monitorare le milestone, visualizzare l'avanzamento delle attività che vengono completate o che sono in fase di sviluppo.

\section{Risorse} % davide
Le risorse relative allo sviluppo software sono le seguenti:
\begin{itemize}
    \item \textbf{Eclipse IDE:} per lo sviluppo e la modifica del codice Java.
    \item \textbf{Maven:} per la gestione e l’automazione dei progetti Java.
    \item \textbf{SQLite:} per la creazione e gestione di un database embedded.
    \item \textbf{Vaadin:} per la progettazione iniziale dell’interfaccia grafica dell’applicazione.
\end{itemize}

\textbf{Strumenti aggiuntivi:}
\begin{itemize}
    \item \textbf{LaTeX:} per la redazione della documentazione tecnica.
    \item \textbf{Papyrus:} per la modellazione di diagrammi UML.
\end{itemize}

Le risorse hardware si limitano ai dispositivi personali dei membri del team.

\section{Budget} % delli
Il budget del progetto, essendo privo di spese finanziarie, è relativo al tempo di lavoro necessario, ipotizzato intorno alle 60 ore per ciascun membro. 

\section{Cambiamenti} %fra
Le modifiche principali saranno apportate durante la fase di ingegneria dei requisiti, per garantire un prodotto il più possibile aderente alle richieste. Eventuali modifiche al codice, derivanti dalla scrittura di test o da cambiamenti delle funzionalità dell'applicazione, verranno gestite attraverso la creazione di un nuovo branch nel repository, accompagnata dall'aggiornamento della documentazione correlata.

\section{Consegna} % davide
La documentazione e l'applicazione finale saranno consegnate condividendo il progetto sul repository GitHub.





\end{document}
