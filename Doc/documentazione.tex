\documentclass[a4paper,12pt]{article}

% Carichiamo il pacchetto per la gestione delle intestazioni, per le immagini e per l'indice
\usepackage{fancyhdr}
\usepackage{graphicx}  % Per inserire le immagini
\usepackage{geometry}
\usepackage{lipsum}
\usepackage{hyperref}  % Per rendere l'indice cliccabile

\geometry{top=2.5cm,bottom=2.5cm,left=2.5cm,right=2.5cm}

% Intestazioni personalizzate
\pagestyle{fancy}
\fancyhf{}  % Pulisce l'intestazione e il piè di pagina
\fancyhead[C]{Progetto di Ingegneria del Software}

\begin{document}

% Titolo
\begin{titlepage}
    \centering
    \vspace*{2cm}
    
    \vspace{1.5cm}
    
    % Titolo del progetto
    \large Documentazione\\
    \Huge
    \textbf{UniMarket}
    
    \vspace{1.5cm}
    
    % Titolo del corso e altri dettagli
    \LARGE
    Progetto di Ingegneria del Software
    
    \vspace{1.5cm}
    \includegraphics[width=0.5\textwidth]{../Media/logo.jpeg}
    
    \vspace{1.5cm}
    % Autori del progetto
    \small
    \textbf{Di Davide Dell'Anno}, \textbf{Davide Carissoni}, \textbf{Francesca Corrente}
    
    \vspace{1.5cm}
    
    % Dati dell'università
    \Large
    Università degli Studi di Bergamo \\
    Facoltà di Ingegneria Informatica \\
    2024-2025
    
\end{titlepage}

% Indice
\renewcommand{\contentsname}{Indice}
\newpage
\tableofcontents
\newpage

% Capitoli e sottosezioni
\section{Software Engineering Management}
\subsection{Project Plan}

\section{Software Life Cycle}
\subsection{Processo di Sviluppo}
\subsection{Documentazione sul Processo}

\section{Configuration Management}
\subsection{Uso di Git}
\subsection{Documentazione su Configuration Management}

\section{People Management and Team Organization}
\subsection{Organizzazione del Lavoro}
\subsection{Documentazione sull'Organizzazione}

\section{Software Quality}
\subsection{Gestione della Qualità}
\subsection{Documentazione sulla Qualità}

\section{Requirement Engineering}
\subsection{Requisiti e Specifiche}
\subsection{Documentazione sui Requisiti}

\section{Modelling}
\subsection{Diagrammi UML}

\section{Software Architecture}
\subsection{Descrizione dell'Architettura}
\subsection{Uso di Maven e Librerie Esterne}

\section{Software Design}
\subsection{Descrizione del Design}
\subsection{Design Pattern e Metriche}

\section{Software Testing}
\subsection{Piano di Test}
\subsection{Casi di Test}
\subsection{Misura della Copertura}

\section{Software Maintenance}
\subsection{Attività di Reverse Engineering}
\subsection{Refactoring}
\subsection{Documentazione sul Refactoring}

\end{document}
