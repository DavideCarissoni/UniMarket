\documentclass[a4paper,12pt]{article}

% caricamento dei diversi pacchetti
\usepackage{fancyhdr}
\usepackage{graphicx}  % Per inserire le immagini
\usepackage{geometry}
\usepackage{lipsum}
\usepackage{hyperref}  % Per rendere l'indice cliccabile

\geometry{top=2.5cm,bottom=2.5cm,left=2.5cm,right=2.5cm}

% Intestazioni personalizzate
\pagestyle{fancy}
\fancyhf{}  % Pulisce l'intestazione e il piè di pagina
\fancyhead[C]{Progetto di Ingegneria del Software}

\begin{document}

% Titolo
\begin{titlepage}
    \centering
    \vspace*{2cm}
    
    \vspace{1.5cm}
    
    % Titolo del progetto
    \large Documentazione\\
    \Huge
    \textbf{UniMarket}
    
    \vspace{1.5cm}
    
    % Titolo del corso e altri dettagli
    \LARGE
    Progetto di Ingegneria del Software
    
    \vspace{1.5cm}
    \includegraphics[width=0.5\textwidth]{../Media/logo.jpeg}
    
    \vspace{1.5cm}
    % Autori del progetto
    \small
    \textbf{Di Davide Dell'Anno}, \textbf{Davide Carissoni}, \textbf{Francesca Corrente}
    
    \vspace{1.5cm}
    
    % Dati dell'università
    \Large
    Università degli Studi di Bergamo \\
    Facoltà di Ingegneria Informatica \\
    2024-2025
    
\end{titlepage}

% Indice
\renewcommand{\contentsname}{Indice}
\newpage
\tableofcontents
\newpage

% Capitoli e sottosezioni
\section{Introduzione} %fra

\section{Modelli di Processo} %davide
Lo sviluppo del progetto seguirà un processo \textbf{RAD} (Rapid Application Development), con l’obiettivo di realizzare un’applicazione funzionante entro i tempi stabiliti dal \textit{time box}.  
Verrà adottato il \textbf{modello di Kano} per classificare i requisiti del progetto in base alla loro capacità di soddisfare le preferenze del cliente.


\section{Organizzazione del Progetto} %delli

\section{Standard, Linee Guida, Procedure} %fra

\section{Attività di Gestione} %davide
Per garantire una gestione efficace delle attività e dello sviluppo del progetto, sono stati definiti i seguenti obiettivi:
\begin{itemize}
    \item \textbf{Incontri settimanali:} monitorare l’avanzamento del lavoro, affrontare eventuali problemi riscontrati, discutere strategie risolutive e valutare la direzione dello sviluppo.
    \item \textbf{Kanban board su GitHub:} tracciare il progresso delle attività, organizzare il lavoro e assegnare incarichi ai membri del team.
    \item \textbf{Gestione del progetto con GitHub:} tenere traccia delle modifiche, monitorare gli aggiornamenti e facilitare la collaborazione.
\end{itemize}

\section{Rischi} %delli

\section{Personale} % fra

\section{Metodi e Tecniche} % davide
I metodi e le tecniche adottati per le diverse fasi dello sviluppo sono i seguenti:
\begin{itemize}
    \item \textbf{Pianificazione dei requisiti:} si basa sull'elicitazione e sull'analisi dei requisiti per definire in modo chiaro le funzionalità richieste dall'applicazione.
    \item \textbf{Progettazione dell’applicazione:} prevede la definizione del funzionamento dell'applicazione, delle funzionalità incluse e degli strumenti software da utilizzare.
    \item \textbf{Programmazione:} consiste nello sviluppo del codice sorgente dell’applicazione seguendo le specifiche progettuali.
    \item \textbf{Testing:} include la scrittura e l’esecuzione di test, utilizzando strumenti appositi, per verificare il corretto funzionamento del codice e individuare eventuali anomalie.
\end{itemize}
Queste attività saranno supportate da un’adeguata gestione del controllo di versione e della configurazione dei componenti software, documentando ogni fase con precisione.


\section{Garanzie di Qualità} % delli

\section{Paccheti di lavoro} % fra

\section{Risorse} % davide
Le risorse relative allo sviluppo software sono le seguenti:
\begin{itemize}
    \item \textbf{Eclipse IDE:} per lo sviluppo e la modifica del codice Java.
    \item \textbf{Maven:} per la gestione e l’automazione dei progetti Java.
    \item \textbf{SQLite:} per la creazione e gestione di un database embedded.
    \item \textbf{Vaadin:} per la progettazione iniziale dell’interfaccia grafica dell’applicazione.
\end{itemize}

\textbf{Strumenti aggiuntivi:}
\begin{itemize}
    \item \textbf{LaTeX:} per la redazione della documentazione tecnica.
    \item \textbf{Papyrus:} per la modellazione di diagrammi UML.
\end{itemize}

Le risorse hardware si limitano ai dispositivi personali dei membri del team.


\section{Budget} % delli

\section{Cambiamenti} %fra

\section{Consegna} % davide
La documentazione e l'applicazione finale saranno consegnate condividendo il progetto sul repository GitHub.





\end{document}
